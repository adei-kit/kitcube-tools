\begin{DoxyVerb}  It is asumed that the PC clock follows
  the unix-like time modell. The function "time"
  will return the UTC time in seconds starting from
  the 1.1.1970. Every 4 years a leap year
  occures but leap second are ignored (see "man 2 time").

  To set the GPS time for the crate two informations are
  required:
\end{DoxyVerb}
 \begin{DoxyItemize}
\item The difference between the time base of U\-T\-C and G\-P\-S time (The G\-P\-S time started at the 6.\-1.\-1980) \item The number of leap seconds since the begin of the G\-P\-S time.\end{DoxyItemize}
The first information does not change, while the number of leap second will increase from time to time. The information when another leap second is added is published some month in advance.

List of leap seconds (form Time\-Stmp.\-cc by T.\-Paul) \begin{DoxyVerb}// Table of GPS seconds when leap second occured.
// THIS HAS TO BE EDITED BY HAND WHENEVER A NEW LEAP SECOND
// OCCURS.  Maybe there is a better way... D.V.: What about putting it
// in an XML file and read it in at singleton runtime initialization so
// that the code does not need recompilation?
//
// Table contains pairs of (GPS second when leap occurred, correction to apply)
//
const unsigned int kNumLeaps=13;
const int kSecPerDay = 24*3600;
const unsigned int kLeapSeconds[kNumLeaps][2] = {
  //
  // (GPS epoch + years + leap days + Jan-Jul)*kSecPerDay + leapSeconds
  //                                           |
  { (361 +       0 +   0 + 181)*kSecPerDay +  0, +1  },    // 1 JUL 1981
  { (361 +     365 +   0 + 181)*kSecPerDay +  1, +2  },    // 1 JUL 1982
  { (361 +   2*365 +   0 + 181)*kSecPerDay +  2, +3  },    // 1 JUL 1983
  { (361 +   4*365 +   1 + 181)*kSecPerDay +  3, +4  },    // 1 JUL 1985
  { (361 +   7*365 +   1      )*kSecPerDay +  4, +5  },    // 1 JAN 1988
  { (361 +   9*365 +   2      )*kSecPerDay +  5, +6  },    // 1 JAN 1990
  { (361 +  10*365 +   2      )*kSecPerDay +  6, +7  },    // 1 JAN 1991
  { (361 +  11*365 +   3 + 181)*kSecPerDay +  7, +8  },    // 1 JUL 1992
  { (361 +  12*365 +   3 + 181)*kSecPerDay +  8, +9  },    // 1 JUL 1993
  { (361 +  13*365 +   3 + 181)*kSecPerDay +  9, +10 },    // 1 JUL 1994
  { (361 +  15*365 +   3      )*kSecPerDay + 10, +11 },    // 1 JAN 1996
  { (361 +  16*365 +   4 + 181)*kSecPerDay + 11, +12 },    // 1 JUL 1997
  { (361 +  18*365 +   4      )*kSecPerDay + 12, +13 },    // 1 JAN 1999
  { (361 +  25*365 +   5      )*kSecPerDay + 13, +14 }     // 1 JAN 2006 // 820022413 , ak
};\end{DoxyVerb}
 \begin{DoxyVerb}What happens with the different clocks when a leap
second occures?
\end{DoxyVerb}


\begin{DoxyVerb}    Civil UTC time       PC wo NTP sync?   PC w NTP sync    Leap     GPS  Leap
    31.12.2005 23:59:58        200             200           32      700   13
    31.12.2005 23:59:59        201             201           32      701   13
     1. 1.2006 23:59:60        202             202           32      702   13
     1. 1.2006 00:00:00        203             202           33      703   14
     1. 1.2006 00:00:01        204             203           33      704   14
\end{DoxyVerb}
 \begin{DoxyVerb}- localtime(<second counter>) will always return 59:59 or 00:00.
  The unix system does not know anything from the doubled second with the
  number 202.
- A PC without NTP synchronisation will return a wrong time by one second.
- The equation NTP = GPS + Offset + LeapSec is always valid.
  The counting of the GPS is unique and linear in time. 
- The feshel function chktime will work as long as the 
  ntp daemon of the PC will increment the PC clock after the leap
  second has occured. 

Dataformat of the Inifile for the leap seconds
\end{DoxyVerb}
 \begin{DoxyVerb}[GPSTime]
; Give a list of leap seconds. The relation between
; NTP time and GPS time is only for the periods
; correct where the leap seconds are given.
;
; Format: <month> <year> <num>
; Where <month> <year> give the date where the number
; of leap seconds changes
; <num> gives the number of leap seconds in the period
; starting with the given date. The leap seconds are
; counted as published in IERS Bulletin C (TAI - UTC/NTP)
;
leap0 = jan 1980 19
leap1 = jul 1997 31
leap2 = jan 1999 32
leap3 = jan 2006 33
\end{DoxyVerb}
 